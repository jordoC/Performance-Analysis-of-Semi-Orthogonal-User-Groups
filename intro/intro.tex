The objective of this paper is to devise a practical method for associating stations (STAs) to access points (APs) for application in IEEE 802.11 MU-MIMO downlink. Currently the 802.11 standard associates STAs with APs on a strongest signal first (SSF) basis \cite{IEEE80211}. While this method of association works well in sparsely populated networks, it is not necessarily the best choice in densely populated networks. Since the population of STAs and APs is unlikely to be uniformly spatially distributed throughout the local area network (LAN), the SSF association scheme is prone to overloading some APs while other APs that are within service range of STAs causing the bottleneck go under utilized.

Contrasting the naive SSF approach, it is theoretically conceivable that a globally optimal association solution in the sense of fairness-weighted sum rate might be found by way of a brute force search. In practice, however, the size of this search space grows very quickly. STAs must first be slotted into concurrent transmission groups (CTGs). Once the set of potential CTGs have been formed, each a set of potential associations is generated by considering each potential CTG associated with each AP. Effectively, this amounts to a cascading the result of one binomial coefficient into the argument of another. Thus, a brute force approach is completely intractable in practice.

The approach taken here is to place constraints on the formation of CTGs. More specifically, the candidate STAs for addition to CTGs are subjected to constraints on channel gain for any given channel in the group and minimum degree orthogonality between channel vectors associated with STAs in the CTG. CTGs formed subject to these constraints are referred to as semi-orthogonal user selection (SUS) groups. Semi-orthogonality has been the subject of various investigations including \cite{Swannack2005}, \cite{Yoo2006}. It has been shown in this work that by subjecting users (STAs in our case) to such norm and orthogonality constraints when forming SUS groups, MU-MIMO sum rate performance using beamforming is asymptotically optimal as the number of users becomes sufficiently large. Thus, we adopt a SUS approach as a method of limiting the search space of the optimal association problem while attempting to minimize the reduction in optimality of this more practical solution.

A spherical packing approach is adopted similar to that proposed by \cite{Swannack2005}. Motivation for using this approach is based on the flexibility of trading off the strictness of the channel norm and orthogonality requirements and probability of finding a SUS group that satisfies these constraints. Such flexibility is of particular interest in our context as it allows us to control the size of the association optimiziation search space as a function of expected SUS group sum rate.

The spherical packing approached developed by \cite{Swannack2005} is found to have limitations in terms of the tightness of SUS existence probability bounds. The looseness of this bound is clearly illustrated in this work by comparing these theoretical lower bounds on probability to numerically simulated values. This result is not unexpected, as the union bounds underpinning this lower bound become increasingly loose in some scenarios including large SUS group sizes and small number of transmit antennas.

The work shows that the looseness of these bounds can be mitigated by making use of widely linear processing operating on purely real transmitted data (ie. single dimensional constellations such as BPSK or M-PAM). It has been shown that such widely linear processing techniques used in conjunction with purely real constellations has (TODO: COMMENT ON MAJID'S RESULTS)