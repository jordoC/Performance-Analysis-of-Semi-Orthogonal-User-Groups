A maximum ratio transmission (MRT) scheme is utilized for determining the transmission beamforming vectors, $\underline w_i,\ i = 1,2,\ldots K$. MRT is based on maximizing the signal-to-noise ratio (SNR)--it neglects interference. The SNR for $STA_i$ is given by:
\begin{equation}
     \begin{aligned}\label{eq:snr}
        SNR_i &= \frac{\vert \underline{h_i}^H \underline w_i \vert^2}{\sigma_n^2}\\
        &= \frac{\vert \underline{h_i}^H \Tilde{\underline w_i} \vert ^2 P_i}{\sigma_n^2}
     \end{aligned}
 \end{equation}
 By the Cauchy-Schwarz inequality, we know that $SNR_i$ is maximized when the channel vector and the beamforming vector are in the same direction, thus maximizing the magnitude of their inner product. Therefore, we let $\underline w_i^{\star} = \underline h_i$ for MRT beamforming. The direction of the MRT beamforming vector becomes $\Tilde{\underline w_i}^{\star} = \frac{\underline{h_i}}{\Vert h_i \Vert}$.
 
 The the upper bound for the SNR, which is achieved by using an MRT beamforming scheme, is given by:
 \begin{equation}
     \begin{aligned}\label{eq:snr_mrt}
            SNR_i^{\star} &= \frac{\vert \underline{h_i}^H \frac{\underline{h_i}} {\Vert \underline{h_i} \Vert} \vert ^2 P_i}{\sigma_n^2}\\
            &= \frac{\Vert \underline{h_i}\Vert^2 P_i}{\sigma_n^2}
     \end{aligned}
 \end{equation}
 
 We now consider interference to develop an expression for SINR that makes use of the MRT scheme. It is important to notice the MRT scheme does not mitigate the effects of interference. The MRT scheme is a special case of a more general beamforming scheme that does not neglect interference (as one might expect). A method for arriving at such a beamforming scheme is discussed at some length in \cite{Bjornson2014}. An MRT scheme is assumed here for the sake of practical simplicity. The SINR is an extension of the SNR given in Eq. (\ref{eq:snr_mrt}) that can be realized by observing the structure of the expanded receive signal given in Eq. (\ref{eq:rx_sig_expanded}).
  \begin{equation}
     \begin{aligned}\label{eq:sinr_mrt}
            SINR_i^{\star} &=  \frac{\Vert \underline{h_i}\Vert^2 P_i}{\sum_{j \neq i}^K \vert \frac{\underline{h_i}^H\underline{h_j}}{\Vert \underline{h_j}\Vert}\vert^2 P_j + \sigma_n^2}
     \end{aligned}
 \end{equation}
 
 Observe from Eq. (\ref{eq:sinr_mrt}) that the interference term in the denominator is in the form of an inner product between two channel vectors. This inner product term can be interpreted as a measure of orthogonality between the two stochastic channel vectors. Consider comparing this term to the orthogonality requirement in Eq. (\ref{eq:S_e}).

The upper and lower bounds on the SNR are used to approximate the power allocated to each beamforming vector. Since the channel norms are bounded, we approximate that each beamforming vector the same amount of power allocated to it. That is, the total power is evenly divided amongst each of the beamforming vectors: $P_i = \frac{P_{tot}}{N}\ \forall i=1,2\ldots K$.

We can make substitutions to Eq. (\ref{eq:sinr_mrt}) that incorporate SUS orthogonality and channel norm requirements from \ref{eq:S_e}.
\begin{equation}\label{eq:sinr_epsilon}
    \begin{aligned}
        SINR_i^{\star} &=  \frac{\Vert \underline{h_i}\Vert^2 \frac{P_{tot}}{N}} {\epsilon^2 \frac{P_{tot}}{N} \sum_{j \neq i}^K  \frac{1}{\Vert \underline{h_j}\Vert^2} + \sigma_n^2}
    \end{aligned}
\end{equation}

It is worth-while reiterating that channel norms follow a Gamma distribution: $\Vert \underline{h_i} \Vert^2 , \Vert \underline{h_j} \Vert^2 \sim \mathcal{G}(2N,\frac{2}{N})$. Therefore, the expression given in Eq. (\ref{eq:sinr_epsilon}) is valid at a given instant in time. The fading channel is assumed to be fast with respect to symbol transmit period. Ergodicity is assumed, and the statistical mean of Shannon capacity is calculated by taking the arithmetic mean with a sufficiently large number of samples and equating it to the statistical mean. There is further elaboration on this point in the experiment implementation discussion dealing with Monte Carlo analysis. The instantaneous Shannon capacity conditioned on SUS group existence is given by:
\begin{equation}\label{eq:capacity}
    \begin{aligned}
        C_i^{\star} \ \vert \ [Pr_\epsilon = 1] &= \log_2(1 + SINR_i^{\star})\ \text{bps/Hz}
    \end{aligned}
\end{equation}

The conditional sum rate for the SUS group is given by:
\begin{equation}\label{eq:sum_rate_conditional}
    \begin{aligned}
        C^{\star} \ \vert \ [Pr_\epsilon = 1] &= \sum_{i=1}^K C_i^{\star}  \ \vert \ [Pr_\epsilon = 1]
    \end{aligned}
\end{equation}

It is important to note here that only one SUS group receives data at any given time. For example, if there are four transmit antennas, and there are two STAs in the group, then only two STAs receive service. Two of the transmit antennas are unused, and only one SUS group of two STAs receives service at a given time. It is also worth while to note that the no weighting scheme that encourages fairness used.

The expression in Eq. (\ref{eq:capacity}) is conditioned on the fact that the SUS group exists. In the context of this experiment, there are a finite number of STAs available to the AP. Moreover, a finite subset of these STAs are considered for addition to a given SUS group in order to limit channel sounding overhead. Therefore, it is not guaranteed that an SUS group will exist. Furthermore, the probability that an SUS group exists, $P_{\epsilon}$, may be expressed in terms of the parameters $\epsilon,\rho^-,\rho_+$, in Eq. (\ref{eq:p_exist}), and the number of candidate STAs considered for addition to the SUS group \cite{Swannack2005}. Thus, the unconditional sum rate for the SUS group is given by:
\begin{equation}\label{eq:sum_rate}
    \begin{aligned}
        C^{\star} &=  (C^{\star} \ \vert \ [Pr_\epsilon = 1])\cdot Pr_\epsilon
    \end{aligned}
\end{equation}






